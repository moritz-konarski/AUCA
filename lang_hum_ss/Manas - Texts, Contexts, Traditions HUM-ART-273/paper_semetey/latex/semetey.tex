% deconstruct the scene of kul-coro and later semetey crossing the river in 
% the two versions and his reception by ay-curok and the girls
% just the section from kul-coro crossing the river until before semetey and 
% ay-curok consummate their marriage
%
%paper for Manas class with J. Plumtree

\documentclass[12pt,a4paper]{article}
\usepackage[margin=1in]{geometry}
\usepackage[utf8]{inputenc}
\usepackage{textcomp}

% commands for names
\newcommand{\Se} {Semetey}               
\newcommand{\Ma} {Manas}            
\newcommand{\Ka} {Kanıkey}              
\newcommand{\Ac} {Ay-čürök}
\newcommand{\Ca} {Čačıkey}
\newcommand{\Ck} {Čın-kojo}
\newcommand{\As} {Ak-šumkar}
\newcommand{\Ak} {Akun Khan}
\newcommand{\Tb} {Tay-buurul}
\newcommand{\Mb} {Maldıbay Borzu uulu}
\newcommand{\Mbs}{\emph{SMB}}
\newcommand{\Tj} {Tınıbek Japıy uulu}           
\newcommand{\Tjs}{\emph{STJ}}
\newcommand{\Dp} {Daniel Prior}
\newcommand{\Kuc}{Kül-čoro}           
\newcommand{\Kac}{Kan-čoro}          
\newcommand{\Dpt}{The Twilight Age of the Kirghiz epic tradition}

\begin{document}

\title{A Hero's Welcome for \Se{}}
\author{Moritz M. Konarski}
\date{15.12.2019}
\maketitle

% what is this paper and what are the texts
This paper will compare the welcome \Se{} receives when meeting \Ac{} in the \emph{\Se{}}
versions by \Mb{} (\Mbs{}) and \Tj{} (\Tjs{}) translated in \Dp{}'s \emph{\Dpt} from 2002.
According to \Dp{}, both texts were written in Kirghiz around the same time --
\Mbs{} in 1899 and \Tjs{} between 1898 and 1902. \Mbs{} contains 6000
lines of poetry split into two \emph{\Se{}} poems that are from 
two different poets. \Tjs{} consists of 3600 lines and was published in 1925 by
the Kirghiz publisher Arabayev. Comparing these two \emph{\Se{}} versions is 
convenient because 
they were written at the same time and they share an almost identical plot. 
To compare the welcome \Se{} receives in both versions, firstly, the plot 
of the two versions will be summarized. Secondly, the welcome \Se{} receives in both
versions will be compared.

% plot summary before event to be discussed 
The story begins with the protagonist \Se{} riding out with his two companions \Kuc{} 
and \Kac{} to retrieve his gyrfalcon \As{}. The gyrfalcon was stolen by the
shape-shifting woman \Ac{} to 
lure \Se{} away from his wife \Ca{}. \Ac{} is in danger to be taken 
as booty by \Ck{}, who is besieging her father \Ak{}'s city to take her as his
wife. She wants \Se{} to come after her to get his gyrfalcon back and hopes he
will break the siege in the process.
When \Se{} and his companions get close to \Ak{}'s city they rest near
a mountain; \Se{} climbs it and looks ahead with his telescope. 
He sees \Ac{} and her maids outside of the city and sends his companion \Kuc{}
to see if they are who he thinks they are. \Kuc{} takes
\Se{}'s armor, weapons, and horse \Tb{} (to look good and travel safely) and leaves. 
Separating \Se{} and his companions from the
maids is a raging river that they have to cross to reach them.
\Kuc{} gets to the river and cannot turn 
back in front of the maids because that would be shameful and crosses it,
albeit with some trouble. 
When \Se{} follows \Kuc{} and gets to the river, \Kac{} suggests to look for a 
ford or stay for the day. \Se{} cannot take the shame of not crossing, especially 
since \Kuc{} successfully crossed before him. Thus he decides to plunge into the river, 
taking all the horses with him. After almost drowning -- \Kuc{} took his 
good horse -- and being saved by ancestral spirits, \Se{} rides up to \Ac{} and 
her maids to speak to them. 

% specifics of SMB
In the \Mbs{} version (Prior, p. 313-315), \Ac{} watches
\Se{} cross the river and says he will not make it. \Kuc{} tells her \Se{} is
strong and protected by spirits, just as he is helped out of the water by them.
\Ac{} then goes to greet \Se{}, addressing him as "my lord" and bringing him
\emph{arak}. Because \Kuc{} has retrieved \Se{}'s gyrfalcon \As{}, \Se{} wants to
leave as that is what he came for. \Ac{} explains to him that they were
betrothed by their fathers before they were born and that she has been waiting
for him to find her. They make a plan to go to \Ac{}'s father's city to be safe from
her suitor \Ck{}, ready a large yurt, and \Se{} and \Ac{} go to bed
together. 

% specifics of STJ
The \Tjs{} version of these events (Prior, p. 330-332), \Ac{} and
her maids laugh at \Se{}, who almost drowned, "until they wet their pants."
Then, \Se{} scolds \Kuc{} for not inviting \Ac{} to his lands and 
not coming to greet him. \Ac{} tells him that, regardless of what he may say,
she has a man (\Ck{}). Angrily, \Se{} demands his gyrfalcon back and asks why he would 
even marry her, as she is old and has white hair. \Ac{} tells him, as in
\Mbs{}, that their fathers betrothed them before their birth and she waited for
\Se{} because she honors them. If he wants \As{} back, he will have to fight
her for it. One of \Ac{}'s maids intervenes, calms her down, and tells
her to serve \Se{} \emph{arak}. She does this and together with \Se{} goes to
her father \Ak{}'s city because out on the steppe \Ck{} might attack them.

% similarities of these two versions
Both \emph{\Se{}} versions have elements in common. \Se{} and \Ac{}
start off at odds with each other because she stole his gyrfalcon, he 
wants to get it back and leave. \Ac{}, as she had a plan when stealing the
bird, tells him about their betrothal. \Se{} gets served \emph{arak} in both
stories, he and \Ac{} consummate their betrothal, and they decide to go or go 
to \Ak{}'s city for safety from \Ck{}. The story they tell has the same beginning 
and arrives at the same conclusion. These parallels between the two versions
suggest that they both originated from a single version of this story. The fact
that they were written at about the same time supports this theory. These two
versions could show the core of this \emph{\Se{}} narrative around 1900. 

% differences between the versions
The way this \emph{\Se{}} narrative is delivered in \Mbs{} and \Tjs{} is very 
different though. In the former, \Se{} is treated like a Khan or hero would
expect to be treated. His companion \Kuc{} believes in him surviving the river
crossing, \Ac{} greets him, calls him "lord" and offers him a drink. She then
simply explains the situation around their betrothal to \Se{}. In \Tjs{} this
is handled very differently. \Ac{} and her maids mock \Se{} after his river
crossing, \Ac{} rejects him at first because she has a suitor, \Se{} is
insulted and insults her back, \Ac{} explains their betrothal, challenges \Se{}
to fight her and needs to be calmed down by one of her maids. Then 
she offers \Se{} \emph{arak}. 

In \Tjs{} \Se{} is not treated like one would
expect a hero or protagonist to be treated. He is mocked, rejected, and
challenged by a woman (not expected in the context of a male-dominated epic).
The difference in delivery of what is essentially the same story might, at
least in part, be the result of the bard's preferences and style. In oral
epics,
each bard tells the story in their own way and they often modify them to fit
current circumstances. Considering that the \Tjs{} variant was recorded when
\Tj{} was ordered to by a police chief and that the recording took place at
a police station, this might explain part of the difference. These
circumstances were probably far from ideal and may have led the bard to perform
the epic the way he did. Because the origin of \Mbs{} is not known, nothing can
be said about its inception.
% ===============this part is done=========================

% conclusion
The two versions of \emph{\Se{}} compared in this paper
illustrate how a story can be told in different ways and leave 
different impressions. The \Mbs{} version lets the characters seem mature and 
friendly as they treat each other with respect. In \Tjs{}, the characters 
are immature because they mock each other.
This changes the way the story is perceived. In \Mbs{}, the characters
may not be perfect, but they are reasonable for who they are, while in \Tjs{}
it does not fit that the daughter of a Khan and her maids would be this
rude to \Se{}, someone of high social rank. Unfortunately, the reason for why \Tj{}
presents this \emph{\Se{}} in such a different way will probably remain
a mystery.

\end{document}
