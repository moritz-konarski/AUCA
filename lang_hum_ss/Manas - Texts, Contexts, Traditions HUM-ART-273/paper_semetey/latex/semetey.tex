% deconstruct the scene of kul-coro and later semetey crossing the river in 
% the two versions and his reception by ay-curok and the girls
% just the section from kul-coro crossing the river until before semetey and 
% ay-curok consummate their marriage
%
%paper for Manas class with J. Plumtree

\documentclass[12pt,a4paper]{article}
\usepackage[margin=1in]{geometry}
\usepackage[utf8]{inputenc}
\usepackage{textcomp}

% commands for names
\newcommand{\Se} {Semetey}               
\newcommand{\Ma} {Manas}            
\newcommand{\Ka} {Kanıkey}              
\newcommand{\Ac} {Ay-čürök}
\newcommand{\Ca} {Čačıkey}
\newcommand{\Ck} {Čın-kojo}
\newcommand{\As} {Ak-šumkar}
\newcommand{\Ak} {Akun Khan}
\newcommand{\Tb} {Tay-buurul}
\newcommand{\Mb} {Maldıbay Borzu uulu}
\newcommand{\Mbs}{\emph{SMB}}
\newcommand{\Tj} {Tınıbek Japıy uulu}           
\newcommand{\Tjs}{\emph{STJ}}
\newcommand{\Dp} {Daniel Prior}
\newcommand{\Kuc}{Kül-čoro}           
\newcommand{\Kac}{Kan-čoro}          
\newcommand{\Dpt}{The Twilight Age of the Kirghiz epic tradition}

\begin{document}

\title{A Hero's Reception in \emph{Semetey}}
\author{Moritz M. Konarski}
\date{15.12.2019}
\maketitle

% introduction paragraph
This paper will compare the use of comedy in the \emph{\Se{}}
versions by \Mb{} (\Mbs{}) and \Tj{} (\Tjs{}) translated in \Dp{}'s \emph{\Dpt} from 2002.
According to \Dp{}, both texts were written in Kirghiz around the same time --
\Mbs{} in 1899 and \Tjs{} between 1898 and 1902. \Mbs{} contains 6000
lines of poetry split into two \emph{\Se{}} poems that, according to \Dp{}, are from 
two different poets. \Tjs{} consists of 3600 lines and was published in 1925 by
Arabayev as the first part of a series of stories that was never continued. It was 
recorded when the police chief of Narın ordered \Tj{} to perform the epic at
a police station.

% paragraph purpose
Comparing these \emph{\Se{}} versions is convenient because they were written at 
the same time and they share an almost identical plot. To investigate the use 
of comedy in both versions, firstly, the plot 
of the two versions will be summarized. Secondly, the use of comedy in both
versions will be compared. Lastly, the effects of that comedy and its influence 
on the characters will be discussed.

% a summary of the plot that both versions share as well as a short description
% of each of the texts
The plot of \Mbs{} and \Tjs{} is not identical, but the second poem in \Mbs{}
is very similar to \Tjs{}. The fundamental plot, disregarding differences in
dialog and locations, is the same. 
The story begins with the protagonist \Se{} riding out with his two companions \Kuc{} 
and \Kac{} to retrieve his gyrfalcon \As{}. The gyrfalcon was stolen by the
shape-shifting woman \Ac{} to 
lure \Se{} away from his wife \Ca{}. \Ac{} is in danger to be taken 
as booty by \Ck{}, who is besieging her father \Ak{}'s city to take her as his
wife. She wants \Se{} to come get his gyrfalcon back and break the siege in the
process.

% paragraph purpose
After a long journey, \Se{} and his companions are close to \Ak{}'s city rest
near a mountain.
\Se{} climbs the mountain to look ahead with his telescope and
sees the city of \Ak{} as well as the army of \Ck{}. He also spies \Ac{} and her maids
outside of the city. \Kuc{} volunteers to go and see who is approaching but asks 
for \Se{}'s armor, weapons, and horse \Tb{} so he can make a good
impression and travel safely. On his way to the strangers, \Kuc{} needs to cross
a dangerous river and only makes it because of his excellent horse and
the help of ancestral spirits. He goes to greet the maids, gives them his and \Se{}'s
lineage and boasts of their deeds.

% paragraph purpose
From the mountain \Se{} sees \Kuc{} interacting with the maids and takes \Kac{}
and their horses down to join them. When \Se{} attempts to cross the river, he
barely makes it across because he is not riding \Tb{} and only by calling on
spirits for help he makes it. He is then greeted by the maids and \Ac{} and
learns that he and \Ac{} had been betrothed by their fathers before
birth. The story ends with \Se{} and \Ac{} consummating their union.

% comparing the comedy in each of the texts
While the second poem in \Mbs{} and \Tjs{} tell the same fundamental story, 
the former contains almost no
comedy. In \Mbs{} \Kac{} and \Kuc{} fall asleep drunk on \emph{arak} 
when \Se{} told them to keep watch. Atop the mountain, when \Se{} sends \Kuc{}
to go to the girls \Se{} tells \Kuc{} how to ride \Tb{}, but he responds that 
he knows \Se{}'s horses better than him because he used to take care of them.
When \Kuc{} is talking to \Ac{} \Se{} believes them to be kissing and rides
down to them as a result. None of these passages are real jokes, they just make
\Se{} look like a fool or bad leader, they are not really mocking him either.
It just exemplifies that he not as good of a leader as \Ma{} was.

% paragraph purpose
In comparison, \Tjs{} is full of mockery, mostly about \Se{}. At the beginning
of the story when \Ac{} steals his gyrfalcon, he fails to pick up a piece of cloth from 
horseback and his companions have to help him. \Se{} wonders if someone richer
than him, maybe a rival hero, lost the cloth. When \Kuc{} is sent to \Ac{}'s
party to investigate, he
is challenged by one of the girls and threatened with death if he approaches.
\Kuc{} challenges her back, exposes the breast of another girl, and demands to
know why \Ac{} took \As{}. \Ac{} threatens to have him killed and demands to 
know his lineage. \Kuc{} tells her and again threatens her, this time with
raiding and kidnapping -- the girls laugh at him in return. 
When \Se{} decides to join them, he faces the river and is too
prideful to rest and look for a chance to cross, so he plunges into the water.
Because he gave his good horse \Tb{} to \Kuc{} he nearly drowns crossing
the river. Watching all of this, the girls make fun of him and
laugh until they wet their pants. \Ac{} and \Se{} insult each other and she
even dares him to fight her. After calming down they decide to go to \Ak{}'s city to
feast and rest, where \Ac{} and \Se{} go to bed together. She leaves in the
night and when \Se{} and \Kuc{} find her in her father's yurt, she tells him
off for his bad manners and sends him back to his wife \Ca{}.

% paragraph purpose
In \Tjs{} \Se{} is not only protrayed as a bad leader he is also mocked
continually. This is strange because he is the protagonist of this story and
those are generally not mocked to hell and back. This also does not really fit
a narrative of learning from other's mistakes because 

% trying to make sense of this change and what it could mean
In \Mbs{} \Se{} is more of a failed hero, in \Tjs{} this idea is taken over the
top and he is just made fun of constantly

The constant comedy and failure makes \Se{} look like a fool instead of a hero,
you just cannot take him seriously.

In \Mbs{} he is not a great leader, but at least he is not he butt of every
joke.

There is no real element of "this character is stupid and so you should learn
something from his mistakes" -- it's more like a mock-epic to quote \Dp{} 

Is this in some way related to the way \Tjs{} came to be, was \Tj{} just pissed
off at the police chief and gave him a mocking version of \emph{\Se{}}? 

Because this excerpt got published, at least Arabayev saw something in it that
was worth sharing and maybe even potentially successful. 

\end{document}
