% Midterm report paper Moritz Konarski -- The 3x+1 Problem
\documentclass[12pt, a4paper, reqno]{amsart}

% included packages
\usepackage{amsfonts}
\usepackage{amsmath}
\usepackage{anysize}
\usepackage{graphicx}
\usepackage[utf8]{inputenc}
\usepackage[hidelinks]{hyperref}
\hypersetup{
    colorlinks=true,
    linktoc=all,
    allcolors=black,
    final
}
% set line spacing
\renewcommand{\baselinestretch}{1.3}
% set margin size
\marginsize{1in}{1in}{1in}{1in}

\title{The $3x+1$ Problem}
\author{Moritz Konarski}
\date{27.02.2020}

\begin{document}

\begin{abstract}
    This paper gives an overview of the Collatz function and conjecture.
    Furthermore, its history and some interesting attributes are discussed.
\end{abstract}

\maketitle
\tableofcontents

\section{Introduction}

\begin{itemize}
    \item The $3x+1$ Problem and Collatz Conjecture
    \item  What Makes This Problem Interesting?
    \item  History of the Collatz Conjecture
    \item  Interesting Attributes of the $3x+1$ Problem
    \begin{itemize}
        \item  Cycles of the Function
        \item  Stochastic Approximations
        \item  Stopping Time of the Function
    \end{itemize}
\end{itemize}

\subsection{What is the $3x+1$ Problem?}

\subsubsection{The Function}

Based on the Collatz function $^{[3]}$
\begin{equation}
C(x)= \left\{
    \begin{array}{ll}
        3x+1 \quad &\text{if } x \equiv 1 \text{ (mod 2),} \\
        x/2 \quad &\text{if } x \equiv 0 \text{ (mod 2).}
    \end{array}
\right.
\end{equation}
Is equivalent to the $3x+1$ function $^{[3]}$
\begin{equation}
T(x)= \left\{
    \begin{array}{ll}
        (3x+1)/2 \quad &\text{if } x \equiv 1 \text{ (mod 2),} \\
        x/2 \quad &\text{if } x \equiv 0 \text{ (mod 2).}
    \end{array}
\right.
\end{equation}

\subsubsection{Details}

\begin{itemize}
    \item it is conjectured that for some $x,k \in \mathbb{N} + 1$ we attain 
        $T^{(k)}(x)=1$ $^{[1]}$
    \item the $3x+1$ function $T(x)$ maps 
        $\mathbb{N} + 1 \rightarrow \mathbb{N} + 1$ $^{[4]}$
    \item the function has a \emph{stopping time}, \emph{total stopping time}, 
        and a \emph{trajectory} for each $m$
\end{itemize}

\subsubsection{Stopping Time for $x$}

\begin{itemize}
    \item check that every positive integer up to $x - 1$ iterates to one $^{[1]}$
    \item if $T^{(k)}(x) < x$, we know it will iterate to 1
    \item thus the stopping time is 
        \begin{equation}
            \sigma(x)=\inf\{k:T^{(k)}(x) < x\}
        \end{equation}
\end{itemize}

\subsubsection{Total Stopping Time for $x$}

Total stopping time is the number of steps needed to iterate to 1 $^{[1]}$
\begin{equation}
    \sigma_{\infty}(x)=\inf\{k:T^{(k)}(x)=1\}
\end{equation}

\subsubsection{Trajectory of $x$ Under $T$}

Also called the \emph{forward orbit} of $x$ under $T$, defined as the sequence 
of it forward iterates $^{[3]}$
\begin{equation}
    \{x, T(x), T^{(2)}(x), T^{(3)}(x),\dots\}
\end{equation}

\subsection{The Collatz Conjecture}

\subsubsection{Possible behaviors of $T$}

\begin{enumerate}
    \item the trivial cycle $\{4,2,1,4,2,1,\dots\}$ (reaching 1)
    \item a non-trivial cycle
    \item infinity, having a divergent orbit $^{[1]}$
\end{enumerate}

\subsubsection{The Conjecture}

\begin{itemize}
    \item beginning at any positive integer $x$, iterations of $T(x)$ will 
        eventually reach 1 and enter the trivial cycle $^{[3]}$
    \item equivalent to stating that the total stopping time 
        $\sigma_{\infty}(x)$ are finite $^{[1]}$
    \item if a trajectory of $T(x)$ does \textit{not} contain 1 it is infinite $^{[2]}$
\end{itemize}

\subsection{What Makes This Problem Interesting?}

\begin{quote}Mathematics is not ready for such problems. --- Paul Erdös $^{[1]}$\end{quote}
\begin{itemize}
    \item the problem itself is not important, it has no immediate applications
    \item represents a class of iterative mappings that are interesting
    \item it is simple to state but hard to prove
    \item part of the difficulty comes from its pseudorandom nature of 
        iterations of $T(x)$ $^{[3]}$
\end{itemize}

\section{History of the Collatz Conjecture}

\subsection{Beginnings}

\begin{itemize}
    \item also known as Syracuse Problem, Hasse's Algorithm, Kakutani's 
        Problem, and Ulam's Problem after other people that studied it
    \item named after Lothar Collatz who formulated similar problems in the 
        1930s
    \item academic publishing about it began in the 1970s $^{[3]}$
\end{itemize}

\subsection{Recent Developments}

\begin{itemize}
    \item $>10^{20}$ numbers have been verified to fit the conjecture $^{[4]}$
    \item a September 2019 paper by Terence Tao "Almost All Orbits of the 
        Collatz Map Attain Almost Bounded Values" made progress
    \item research is still actively ongoing
\end{itemize}

\section{Interesting Attributes of the $3x+1$ Problem}

\subsection{Cycles of the Function}

\begin{itemize}
    \item the $3x+1$ function has a trivial cycle $\{4,2,1,4,2,\dots\}$ at 1 $^{[1]}$
    \item if $T(x)$ is applied to all integers, three more cycles emerge at -1, -5, and -17
    \item these cycles are conjectured to be the only ones $^{[1]}$
    \item if non-trivial cycles of the $3x+1$ problem exist, they have been 
        proven to be at least 10,439,860,591 numbers long $^{[3]}$
\end{itemize}

\subsection{Stochastic Approximations}

\begin{itemize}
    \item number of odd and even integers in an orbit is approximately equal
    \item behavior is seen as pseudorandom, if the numbers are large enough 
        they almost behave like random variables $^{[2]}$
    \item probabilistic models describe the behavior of the $3x+1$ problem
    \item models describe groups of trajectories, not individual ones $^{[3]}$
\end{itemize}

\subsection{Stopping Time of the Function}

\begin{itemize}
    \item stopping time for odd numbers is $\approx 9.477955$ for $C(x)$ $^{[1]}$
    \item total stopping time for most trajectories is about $6.95212 \log n$ steps
    \item number of even integers in an orbit equal to stopping time
    \item upper bound for total stopping time $41.677647 \log n$, suggests all
        sequences are finite $^{[3]}$
\end{itemize}

\subsection{Conclusion}

\subsubsection{The $3x+1$ Problem and Collatz Conjecture}

For every $x \in \mathbb{N} + 1$ and the function
\begin{equation}
T(x)= \left\{
    \begin{array}{ll}
        (3x+1)/2 \quad &\text{if } x \equiv 1 \text{ (mod 2),} \\
        x/2 \quad &\text{if } x \equiv 0 \text{ (mod 2).}
    \end{array}
\right.
\end{equation}
there is some $k \in \mathbb{N} + 1$ such that $T^{(k)}(x)=1$.

\subsubsection{What Makes This Problem Interesting?}

\begin{itemize}
    \item simple to state but hard to prove
    \item represents a class of iterative mappings that are interesting $^{[3]}$
    \item maybe mathematics right now cannot solve that problem
\end{itemize}

\subsubsection{History of the Collatz Conjecture}

\begin{itemize}
    \item named after Lothar Collatz, from the 1930s
    \item academic publishing began in the 1970s $^{[3]}$
    \item $>10^{20}$ numbers have been verified to fit the conjecture $^{[4]}$
    \item research is still actively ongoing
\end{itemize}

\subsubsection{Interesting Attributes of the $3x+1$ Problem}

\begin{itemize}
    \item the $3x+1$ function has a trivial cycle $\{2,1,2,\dots\}$ at 1 $^{[1]}$
    \item non-trivial cycles of the $3x+1$ problem have been proven to
        be at least 10,439,860,591 numbers long $^{[3]}$
    \item behavior is seen as pseudorandom, if the numbers are large enough they almost
        behave like random variables $^{[2]}$
    \item total stopping time for most trajectories is about $6.95212 \log n$ steps
\end{itemize}


\begin{equation}
    \pi(x) \approx \rho \times \dagger c^2
\end{equation}

The above equation eq01 refers to some mathematical concept that
I have made up.






% define the elements of the bibliography
% TODO: fix references
\begin{thebibliography}{4}

\bibitem{chamberland} Marc Chamberland, \textit{An Update on the $3x+1$ Problem},
   % \texttt{http://www.math.grinnell.edu/{\textasciitilde{}}chamberl/papers/3x_survey_eng.pdf},
    2005.

\bibitem{crandall} R. E. Crandall, \textit{On the "$3x+1$" Problem}, 
    Mathematics of Computation, \textbf{32} (1978), no. 144, 1281-1292

\bibitem{lagarias} Jeffrey C.Lagarias, \textit{The $3x+1$ Problem: An Overview},
    \\\texttt{https://pdfs.semanticscholar.org/100046dd8b4ee901bc71043da7d42f5d87ca0224.pdf}, 
    2010

\bibitem{tao} Terence Tao, \textit{Almost All Orbits of the Collatz Map Attain 
    Almost Bounded Values}, arXiv:1909.03562v2 [math.PR], 2019
        
\end{thebibliography}
\end{document}
