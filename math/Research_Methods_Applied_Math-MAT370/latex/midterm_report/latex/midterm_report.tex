% Midterm report paper Moritz Konarski -- The 3x+1 Problem
\documentclass[12pt, a4paper, reqno]{amsart}

% included packages
\usepackage{amsfonts}
\usepackage{amsmath}
\usepackage{anysize}
\usepackage{graphicx}
\usepackage[hidelinks]{hyperref}
\hypersetup{
    colorlinks=true,
    linktoc=all,
    allcolors=black,
    final
}
% set line spacing
\renewcommand{\baselinestretch}{1.3}
% set margin size
\marginsize{1in}{1in}{1in}{1in}

\title{The $3x+1$ Problem}
\author{Moritz M. Konarski}
\date{27.02.2020}

\begin{document}

\begin{abstract}
    % TODO: write abstract
\end{abstract}

\maketitle
\tableofcontents

\section{Introduction}

\begin{equation}
    \pi(x) \approx \rho \times \dagger c^2 \label{eq01}
\end{equation}

The above equation \eqref{eq01} refers to some mathematical concept that
I have made up.


















% define the elements of the bibliography
% TODO: fix references
\begin{thebibliography}{2}
    \bibitem{chamberland}
        M. Chamberland. \textit{An Update on the 3x+1 Problem}. 

    \bibitem{crandall}
        R. E. Crandall. \textit{On the "$3x+1$" Problem}. 

    \bibitem{garner}
        L. E. Garner. \textit{On the Collatz $3n+1$ Algorithm}.

    \bibitem{lagaris}
        J. C. Lagaris. \textit{The $3x+1$ Problem: An Overview}
        
\end{thebibliography}
\end{document}
